\documentclass[margin,line]{resume}
\usepackage[hidelinks]{hyperref}

\begin{document}
\name{\Large Samuel Brotherton}
\begin{resume}
    \section{\mysidestyle Contact\\Information}

    Phone: (801) 927-0219       \hfill LinkedIn: \url{https://www.linkedin.com/in/sbrother} \\
    \noindent Email: sbrother@gmail.com  \hfill Address: 7771 Buckboard Drive, Park City, UT 84098 \vspace{0mm}\\\vspace{-4.5mm}

    \section{\mysidestyle Education}

    \textbf{Harvard University}, Cambridge, MA \vspace{2mm}\\\vspace{1mm}%
    \textsl{B.A., Mathematics and East Asian Studies} \hfill \textbf{Sep 2008 -- May 2012}\\
    Received highest honors for senior thesis analyzing over 200,000 Chinese
    blog posts, algorithmically detecting mutations in the Chinese language in
    response to censorship. Completed coursework in abstract algebra,
    Galois theory, topology, real and complex analysis, probability theory, and
    linguistics.

    \section{\mysidestyle Professional\\Experience}

    \textbf{Google}, Los Angeles, CA / Park City, UT (Remote)  \vspace{2mm}\\\vspace{1mm}%
    \textsl{Software Engineer} \hfill \textbf{Feb 2021 -- Present}\\
    Software engineer on an infrastructure team within Ads Privacy and Security,
    using natural language processing and machine learning to prevent
    policy-violating ad impressions. Our systems handle 2M+ queries per second
    and combine modern deep learning techniques with multi-tier human
    evaluation pipelines. Additionally, I have a 20\% role building
    remote sensing infrastructure and machine learning models for a stealth
    climate ``startup'' within Google.

    \textbf{Pachama}, San Francisco, CA / Park City, UT (Remote)  \vspace{2mm}\\\vspace{1mm}%
    \textsl{Software Engineer} \hfill \textbf{Jul 2020 -- Feb 2021}\\
    Worked on a small team to productionize Pachama's remote sensing
    models using Python, Tensorflow, Kubernetes, Cloud Dataflow, and other GCP
    technologies. Designed and built a data ingestion service to retrieve,
    transform and cache satellite imagery and LiDAR point clouds for easy use in
    both model training and production. Left to rejoin Google under their new
    remote work policy.

    \textbf{Cairn Labs}, Park City, UT \vspace{2mm}\\\vspace{1mm}%
    \textsl{President and Founder} \hfill \textbf{Jan 2016 -- Mar 2020}\\
    Lead a cross-functional team of 4-6 to design and build software
    for clients, with a focus on applications that integrate
    state-of-the-art NLP and machine learning. Notable projects/clients include:

    \begin{itemize}
    \item Lead a remote (US-based) engineering team to build AI-powered audiovisual
      experiences, including a cloud deep learning server for song generation, a
      C++ constraint/transformation library for imposing additional musical style
      on deep learning output, and an automated Max/Ableton session for audio
      generation. Coordinated with creative and executive teams to balance
      engineering and IP development with artistic and business goals. See
      \url{warpsound.ai} for examples.
    \item A deep learning based conversational UI framework to power
      will.i.am's wireless earphones and other applications. Backed by Python,
      Tensorflow, Prolog, and other technologies. Supports multiple languages,
      extensible dialogue flows, and includes a type system that integrates with
      a custom knowledge base. Our software had a significant role in the
      success of the client's \$117M fundraising round (\url{https://goo.gl/iMvVTt}).
    \item An offline-first, mobile first Electronic Health Records system
      designed for the unique challenges of mobile refugee health clinics.
      Currently deployed in Lebanon and Nicaragua (see Brotherton et al in
      \emph{Selected Publications}).
    \item An intelligent agent for a healthcare client that performs realtime
      conversational analysis from raw audio, running a series of clinically
      validated health classifiers on audio and text from realtime transcription
      (see Demiris et al in \emph{Selected Publications}).
    \item A risk scoring server for a client in the car insurance industry,
      collecting realtime driving data and assigning machine-learning based
      driver risk scores at a rate of 3000+ qps (see \url{motioninsurance.com}).
    \item An advertising server to display appropriate event ticket offers
      to users, based on user-level and page-level contextual targeting. Backed
      by Elixir/Phoenix, Python, PostGIS, and React.js.
    \end{itemize}

    \pagebreak

    \textbf{Google}, Venice, CA \vspace{2mm}\\\vspace{1mm}%
    \textsl{Software Engineer} \hfill \textbf{Jun 2014 -- Mar 2016}\\
    Worked on a small team using natural language processing and other machine
    learning techniques to improve advertisement quality. Led a 20\% project
    related to mining semantic information from web data, which was adopted by
    several teams across different product areas. Built a named entity
    recognition system in C++ and a link detection algorithm that runs on very
    large graphs; contributed to a topic model for clustering semantic entities.

    \textbf{Whisper}, Venice, CA \vspace{2mm}\\\vspace{1mm}%
    \textsl{Software Engineer and Data Scientist} \hfill \textbf{Mar 2013 -- Apr 2014}\\
    Sole data scientist at a rapidly expanding social media startup
    seeing upwards of three billion monthly pageviews. Designed and built an NLP
    service to extract topics and tags from posts, predict image searchterms
    from unstructured text, and target content to users. Implemented a new
    geographic search system using PostGIS that decreased search time by
    90\%. Worked closely with the front and backend development teams, writing
    production code in Erlang and Python.

    \section{\mysidestyle Selected\\Publications}
    Brotherton, T., Brotherton, S., Ashworth, H., Kadambi, A., Ebrahim, H. and Ebrahim, S., 2022. Development of an Offline, Open-Source, Electronic Health Record System for Refugee Care. \emph{Frontiers in Digital Health}, 4.

    Demiris, G., Oliver, D.P., Washington, K.T., Chadwick, C., Voigt, J.D., Brotherton, S. and Naylor, M.D., 2022. Examining spoken words and acoustic features of therapy sessions to understand family caregivers’ anxiety and quality of life. \emph{International Journal of Medical Informatics}, 160, p.104716.

    Demiris, G., Corey Magan, K.L., Parker Oliver, D., Washington, K.T., Chadwick, C., Voigt, J.D., Brotherton, S. and Naylor, M.D., 2020. Spoken words as biomarkers: using machine learning to gain insight into communication as a predictor of anxiety. \emph{Journal of the American Medical Informatics Association}, 27(6), pp.929-933.

    Stephens, M., Bensink, M., Brotherton, S., Chandler, D., Garcia, J. and Hollenbeak, C., 2016. Geographic Access to Oncology Services in the United States (US): Travel Disparities May Affect Granulocyte-Colony Stimulating Factor (G-CSF) Administration. \emph{Blood}, 128(22), p.5905.

    Stephens, M., Brotherton, S., Dunning, S., Emerson, L., Gilbertson, D., Harrison, D.J., Kochevar, J., McClellan, A., McClellan, W., Wan, S. and Gitlin, M., 2011. The End-stage Renal Disease (ESRD) Prospective Payment System (PPS) and Access to Care: Incremental Distance Traveled by Displaced Patients.

\section{\mysidestyle Programming\\Experience}

\emph{Languages:} Python, C++, Erlang, Elixir, Mathematica,  C\#, F\#, Bash, C, \LaTeX \\
\emph{Server Technology:} Kubernetes, Docker, GCP, Cassandra, Redis, PostgreSQL, ElasticSearch/Lucene

\end{resume}
\end{document}
